Mining data can always be improved and extended. There are great number of studies that assess the quality and completeness of data mined from project archives, but only in rare cases the data is compared to qualitative evidence\cite{aranda2009secret}\cite{kalliamvakoupromises}. 

This research paper describes an empirical study that determines the usage of IPython Notebook in the scientific software engineering area. It critically analyses the publicly available data coming from GitHub\cite{gitHubWiki} and it creates a model suitable for investigation. If a researcher seeks to see trends of IPython Notebook use, activity of a project, type of languages in projects, number and type of errors in scripts, and others,
the publicly available data can give solid information about the detailed
characteristics of the IPython projects on GitHub. However, creating a framework for synthesizing and structuring information, in order to estimate the usage of the interactive tool, requires some considerations. The research presents assumptions and techniques about repository activity and content, development and collaboration practices in GitHub repositories, and applications of IPython scripts, depending on their content and quality. The paper recommends that researchers, interested in the extracted data domain from this project, need to assess its fit and investigate it before using it for support of their research analysis.

Analysis over repository's activity is encountering challenges with the validity of any study, because of the bias towards personal use - usually, GitHub repositories having small number of contributors are most likely to be inactive. However, IPython supports the reporting of personal ideas and work. This shows that it is not definite that an IPython project is inactive. Therefore, there are different views for the activity of a project and further analysis and relations can be estimated over the idea.

From the analysis explained in subsection \ref{subsec:mining}, we can conclude that identifying activity of a project depends on the recent time period of development, good balance between number of commits and pull requests, and the number of committers and authors to be larger than two. Another aspects that could be considered is the users and their characteristics. Some project's users are not registered and this indicates that the project is used somewhere outside GitHub - e.g. in some other hosting service, such as BitBucket\cite{bitBucket}.

The techniques in subsection \ref{subsect:ipython_sci} are estimating that IPython projects can be clustered into groups depending on: 1) common parts in scripts, 2) type of projects determined from documentation, 3) amount of code and text, 4) quality and accuracy of scripts' results. Their outcomes shows that IPython Notebook is used on GitHub for presenting mainly code visualisations instead of just describing them with words. 

GitHub data is changing over time. Everyday new information is pushed into various repositories or old one is changed - commits and merges. This means that the GitHub API is, also, altering. This one more factor affecting the research data domain and outcomes. Functionality over tracking quality of data over time can be implemented as a future extension of the prototype.

IPython Notebooks is a great support for accuracy and interaction with information. GitHub contains huge amount of IPython projects, their number continues to grow and, as a result, their users are creating and sharing innovative ideas. 

\subsection*{Acknowledgements} I would like to thank my supervisor Dr. Timothy Storer for all of the support, advices, time and contributions to the project that he provide. 