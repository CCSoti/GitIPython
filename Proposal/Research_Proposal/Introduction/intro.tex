briefly explain the context of the project problem

Designed to make data analysis easier to share and reproduce, the IPython notebook is being used increasingly by scientists who want to keep detailed records of their work, devise teaching modules and collaborate with others. Some researchers are even publishing the notebooks to back up their research papers — and Brown, among others,is pushing to use the program as a new form of interactive science publishing. \cite{shen2014interactive}

The IPython notebook addresses both issues
by helping scientists to keep track of their work,
and by making it easy to share and for others
to explore the code. The ‘I’ in IPython refers to
an ‘interactive’ command window that helps
users to run code, access variables, call up data
analysis packages and view plots, while the
Python refers to the popular programming
language that the notebook is based on. (Pérez,
Granger and their colleagues are now moving
the notebook into a project called Jupyter, which
aims to make IPython more compatible with
other languages, including Julia and R).

Applications similar to the IPython notebook
already exist for various programming
languages. Mathematica and Maple — commercial
analysis packages popular among mathematicians
— include notebooks or notebook-like
programs. MATLAB, a commercial package
used heavily in signal processing, engineering
and medical-imaging research, also supports a
notebook application. Each of these notebooks
is specialized for its corresponding proprietary
programming language.
A number of notebooks and notebook-like
programs exist in the open-source world; knitr
works with the R coding language, which is
especially powerful for statistical analysis. And
the Sage mathematical software system, which
is also based on the Python language, supports
its own notebook. Dexy is a notebook-like program
that focuses on helping users to generate
papers and presentations that incorporate prose,
code, figures and other media.
But the IPython notebook has become one
of the most widely adopted programs of its
kind, says Ana Nelson, the creator of Dexy. “So
many people have heard of it who haven’t heard
of any other tool,” she says. Granger and Pérez
do not know how many people have tried their
software, but say that traffic to the website suggests
that roughly 500,000–1.5 million people
actively use the program. Nelson says that it is
the best-designed of the digital notebooks, and
attracts many users because it is free and open
source. The application also benefits from the
popularity of the Python language, which boasts
a robust scientific community that meets for an
annual international conference, and is (relatively)
easy for novice.