
One of the field that we are going to investigate with this project is the usage of one particular tool - IPython notebook. As mentioned in the Problem Statement section \ref{sec:problem}, IPython has become powerful coding tool for various areas in researches, which leads to the questions - is it helpful?; how many journals and explorations are using IPython? 

Created in 2011, by Fernando Pérez, a data researcher at the University of California and computational physicist Brian Granger at California Polytechnic State University, IPython has become an effective tool for numerical analysis and data visualizations.\cite{shen2014interactive} It is a Python environment, which is HTML-based, similar to Mathematica - \textit{"a symbolic mathematical computation program used in many scientific, engineering, mathematical, and computing fields"}. \cite{mathematicaWiki} \cite{mathematicaWolfram}
\cite{johansson2014introduction}

------------------------------------------------------------------------------------------------------

(Pérez,
Granger and their colleagues are now moving
the notebook into a project called Jupyter, which
aims to make IPython more compatible with
other languages, including Julia and R)

The IPython notebook was developed in 2011
by a team of researchers led by Fernando Pérez,
a data scientist at the University of California,
Berkeley, and computational physicist Brian
Granger at California Polytechnic State University
in San Luis Obispo. “We built it by solving
problems that we ourselves had as researchers
and educators,” says Pérez.

Pérez and Granger saw that data scientists
found it hard to share detailed but understandable
descriptions of their raw code that would
allow others to build on their research. That is
partly because many scientists in computationintensive
fields write code in an iterative and
piecemeal fashion as each analysis reveals new
insight and spins off multiple lines of inquiry.
Keeping track of the different versions of code
that produce various figures, and linking those
files with explanatory notes, is a headache.
And what gets published is usually not detailed
enough for the reader to follow up on. “In my
own computational physics work,” says Granger,
“a high-level description of the algorithm that
goes into the paper is light years away from
the details that are written in the code. Without
those details, there is no way that someone
could reproduce it in a reasonable time scale.”

Applications similar to the IPython notebook
already exist for various programming
languages. Mathematica and Maple — commercial
analysis packages popular among mathematicians
— include notebooks or notebook-like
programs. MATLAB, a commercial package
used heavily in signal processing, engineering
and medical-imaging research, also supports a
notebook application. Each of these notebooks
is specialized for its corresponding proprietary
programming language.
A number of notebooks and notebook-like
programs exist in the open-source world; knitr
works with the R coding language, which is
especially powerful for statistical analysis. And
the Sage mathematical software system, which
is also based on the Python language, supports
its own notebook. Dexy is a notebook-like program
that focuses on helping users to generate
papers and presentations that incorporate prose,
code, figures and other media.
But the IPython notebook has become one
of the most widely adopted programs of its
kind, says Ana Nelson, the creator of Dexy. “So
many people have heard of it who haven’t heard
of any other tool,” she says. Granger and Pérez
do not know how many people have tried their
software, but say that traffic to the website suggests
that roughly 500,000–1.5 million people
actively use the program. Nelson says that it is
the best-designed of the digital notebooks, and
attracts many users because it is free and open
source. The application also benefits from the
popularity of the Python language, which boasts
a robust scientific community that meets for an
annual international conference, and is (relatively)
easy for novice.

------------------------------------------------------------------------------------------------------