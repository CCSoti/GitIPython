Replication and reproducibility are playing an important role in the usage of software in scientific researches and they are crucial aspects of the scientific method. The following implications apply for them:

\begin{itemize}
\item Replication - "An author of a scientific paper that involves numerical calculations should be able to
rerun the simulations and replicate the results upon request. Other scientist should also be able to
perform the same calculations and obtain the same results, given the information about the methods
used in a publication."\cite{johansson2014introduction}
\item Reproducibility - "The results obtained from numerical simulations should be reproducible with an
independent implementation of the method, or using a different method altogether."\cite{johansson2014introduction}
\end{itemize}

In order to succeed with this features, we have to consider:

\begin{itemize}
\item To keep track and record of the code version that is generating specific data.
\item To have access to the environment and software that is used.
\item To save and back up code and documentation for future work.
\item To publish code online, so that it can gain interest of other scientists in the code and the presented findings. \cite{johansson2014introduction}
\end{itemize}

Computational science has led to exciting new developments, but the nature of the work has exposed limitations in our ability to evaluate published findings. Reproducibility has the potential to serve as a minimum standard for judging scientific claims when full independent replication of a study is not possible. \cite{peng2011reproducible}

This article considers the obstacles involved in creating reproducible computational research as well as some efforts and approaches to overcome them.\cite{levequereproducible} The
extensive use of computation in scientific discovery
affects the implementation of these standards:
Parameter values, function invocation sequences,
and other computational details are typically
omitted from published articles but are critical
for replicating results or reconciling sets of
independently generated results. Consequently,
researchers from fields as diverse as geoscience,
neuroscience, bioinformatics, applied mathematics,
psychology, and computer science are calling
for data and code to be made available in such a
way that published computational results can be
conveniently reproduced.

The rise of computational science has led to exciting and fast-moving developments in many scientific areas. New technologies, increased computing power, and methodological advances have dramatically improved our ability to collect complex hig-dimensional data (1, 2). Large data sets have led to scientists doing more computation, as well as researchers in computationally oriented fields directly engaging in more science. The availability of large public databases has allowed for researchers to make meaningful scientific contributions without using the traditional tools of a given field. As an example Python \cite{peng2011reproducible}

show how you plan to organize your work, identifying intermediate deliverables and dates.