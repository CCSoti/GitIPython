Computational science has led to exciting new developments, but the nature of the work has exposed limitations in our ability to evaluate published findings. Reproducibility has the potential to serve as a minimum standard for judging scientific claims when full independent replication of a study is not possible. \cite{peng2011reproducible}

This article considers the obstacles involved in creating reproducible computational research as well as some efforts and approaches to overcome them.\cite{levequereproducible} The
extensive use of computation in scientific discovery
affects the implementation of these standards:
Parameter values, function invocation sequences,
and other computational details are typically
omitted from published articles but are critical
for replicating results or reconciling sets of
independently generated results. Consequently,
researchers from fields as diverse as geoscience,
neuroscience, bioinformatics, applied mathematics,
psychology, and computer science are calling
for data and code to be made available in such a
way that published computational results can be
conveniently reproduced.

The rise of computational science has led to exciting and fast-moving developments in many scientific areas. New technologies, increased computing power, and methodological advances have dramatically improved our ability to collect complex hig-dimensional data (1, 2). Large data sets have led to scientists doing more computation, as well as researchers in computationally oriented fields directly engaging in more science. The availability of large public databases has allowed for researchers to make meaningful scientific contributions without using the traditional tools of a given field. As an example Python \cite{peng2011reproducible}