% clearly state the problem to be addressed in your forthcoming project. Explain why it would be worthwhile to solve this problem.

Creating and writing a research is a challenging process. It is extremely difficult for the author to reach and present unique findings or conclusions. If the analysis for the research have been completed, the scientist has to investigate appropriate arguments and evidence for supporting the ideas which he wants to prove with the research. An effective and well conducted research takes long time to be achieved. Recording all of the findings, ideas and approaches, made during the investigation of the research, is an necessary step for publishing and presenting the analysis. Nonetheless, documenting a project or scientific discovery is challenging and it has to be properly presented to the reader so that it would be understandable. Going thought this process, the analyst might find faults, errors or some unnecessary steps done during investigation, and they be the cause of invalid results. Therefore, scientists have found different forms to report project measurements and practices.\cite{holmes2003reworking}

In the past, the analysis in a scientific research were computed quite slowly with a few number of tests for catching bugs, errors and faulty results. The setting up has also causing issues with the output since the researchers were using various tools for achieving different computations. During the investigation of the research, scientists were cautious to write accurate reports for their ideas, findings and results. The best approach for recording the analyses at that time was handwritten lab notebooks. Researches connected with complex computations, such as finance, mathematics and informatics need a lot of support for maintaining huge amount of code and data, which are used for calculations and experiments. Documenting and recording the results will be of use for comparison of the data quality in future researches and for learning from past mistakes and failures.\cite{guo2012burrito}
Here are some of the challenges that scientists have to complete so the research can achieve a success:\cite{guo2012burrito}

\begin{itemize}
\item In the scientific method of a project, testing the hypothesis is one of the most important steps. Therefore researchers have to frequently alter their work and re-execute so that the data will be up-to-date. The issue that occur in this case is forgetting which change generates a particular output. 

\item Accomplishing a research is a time and effort consuming process. The background research is necessary for scientists to understand the problem and to come up with variety of ideas. As in the first point, remembering all of the sources that inspire ones idea is a troublesome task. 

\item There is a struggle to maintain up-to-date notes. Researchers need to be able to go back into old versions of the data, so that they can compare and analyze the approaches that they use. 

\item The results that scientists that need to achieve have to as accurate as possible. They run various programs and software applications, that create data models, which require a lot of complex computations.

\end{itemize}

There a various ways to handle problems like the ones mentioned above. Scientists have found that using handwritten laboratory notebooks helps with tracking the history of the code alternations. Researchers are, also, taking notes by using simple text files or sticky note tools. However, scientists have to take care of the issue with organizing and connecting these electronic notes together with the handwritten report. The main drawback is that, there is no easy way of editing or executing the code. Even with the paper notebook, it is quite challenging to trail past results. Therefore, electronic notebooks were created - they cope with various issues encountered in the data collection of a research and they are fast in complex analysis.

One of the most crucial task in researches is to test how much the collected information is correct and accurate, which can be done with numerical analysis. Nowadays, computational experiments have been done in projects - e.g. software development needs computational analysis by evaluating a prototype on a specific number of users or modelling relations between data items. Theoretical work requires symbolic and numerical support as well - e.g. how much of the sources were able to prove a specific concept. Almost every scientific paper is finding an effective usage of computer science applications.

For many years researchers has tried to gather data in the fastest, most effective and most structured approach. As the boost of software tools usage in scientific researches, investigators are creating more and more numerical analysis and are able to clearly and efficiently organise their notes. Here is one suggested course of actions given by \textit{"An Open Source Framework For Interactive, Collaborative And Reproducible Scientific Computing And Education"}. \cite{perez2013open}.

\paragraph{Phases:}\cite{perez2013open}
\begin{enumerate}
\item Individual exploration - testing ideas, questions or algorithms for a test data set.

\item Collaboration - if the hypothesis appears to be effective, then the investigators finds others collaborators to expand the research.

\item Production-scale execution - manipulating and managing big data domains on clusters, supercomputers or cloud computing software.

\item Publication - results that are presented to colleagues or researchers in the same field of study.

\item Education - continuing the research as part of a teaching program or in a process of more unique results and findings.
 
\end{enumerate}

The life-cycle of the scientific research can not be traversed with only one software tool. Investigators are using a large number of applications that are supporting each of the phases of the analysis. As a consequence, issues are rising - reduced quality, reproducibility and robustness. The scientists are unable to reuse and maintain the code - the same computational analysis might be useful in some other researches and, also, software changes rapidly and needs constant updates. Testing if the results are correct includes testing of all the pieces of code created for the project. Keeping track of the data quality and accuracy requires well structured and written documentation, which is flawless in describing the entire project's workflow. 

% explain about Python in scientific computing - just how it is used, 1 sentence why
% IPython 
% GitHub is widely used - so we will analyzie its results for IPython


Researchers have found a software tool that helps with the analysis over data quality, share and reproducing - IPython notebook. The application is effective in teaching, collaboration with others, combining text and code and recording notes. IPython has been established by scientists as an interactive tool for journals and support for keeping track of thoughts and ideas. In the research \textit{Interactive notebooks: Sharing the code} by \textit{Helen Shen}, IPython has been described to have an \textit{"easy interface"} and it is creating better communication between collaborators.\cite{shen2014interactive}

Python is a high-level language, which is widely used in various software systems and scientific researches. The most important areas, where Python can be practiced, are data gathering, data manipulation, and data visualization and analysis. Coding on Python is extremely valuable for the quick creation of a software prototype or complicated algorithms. It has become powerful tool in the field of scientific computing. Software developers are creating a wide range of Python libraries for easier and cleaner programming.  What makes Python such an approachable language? What are the main properties of Python that makes it a clear and reasonable for scientific computing? 

Through IPython notebook, coders are able to express their ideas by combining code and text at the same time. The interactive IPython command window support users to run code, access variables and data sets, view graphs and charts.\cite{perez2013open}

To sum up, the open source IPython project presents a solution to several issues encountered in researches and it is a single software tool enabling the spanning of the entire life-cycle of computational analysis. 