\subsection{Content}

Software development projects, scientific researches and teaching computing science are difficult to organise, document and present. Reviewing a piece of code or publishing it has always required explanation and so far it has been recorded as a comment in the code or as a text file. This method of reporting analysis and algorithms is ineffective for the productivity and concentration of project's developers. 

Python is a programming language that is known as an easy to learn, code and prominent for data analysis, except that it has an inconvenient shell. Running, testing, debugging and documenting Python scripts has required the usage of various software tools, such as the command prompt (or another command-line interpreters), text editors and others. 

To overcome this drawback of Python, IPython has been created as a software tool, interactive shell for standard scientific Python scripts, that allows combination of styled text, code and data visualizations. It is designed to stimulate the writing, testing and debugging of Python code and it has a productive environment for analytical computing. \cite{mckinney2012python} Running Python scripts is effective for immediate response, since developers can see the results right away.

how IPython is used in researches
% Is IPython crucial and compelling software tool for researches and journals? The answer of this question can be achieved by analyzing data taken from GitHub - Web-based Git repository hosting service. \cite{gitHubWiki} GitHub is one of the most crucial web-sites as a source of information  on the Internet. A huge number of researchers have started to investigate and anlayze GitHub repostories so that they can understand how users are exploiting the site for collaboration on software. \cite{kalliamvakou2007promises}

\subsection{Problem Statement}
\label{subsec:problem}

Given: IPython in researches, real world problems

Why: to understand what role it takes in these researches

software in science --> found that lab notebooks help --> IPython

Normally, theoretical and experimental practices are representing two main aspects in the process of a research. Recently, computing is appearing to be one more important part of science. It is not only closely connected to theory and experiment, but it also has common features with them. Scientific programming is a crucial element of research analysis, since it gives faster and more accurate results - all calculations, models and visualizations can be computed with the help of software tools. As a consequence, scientific programming can be viewed as a another section of conducting researches.\cite{johansson2014introduction}

Computational sciences bring great help and support to the field of journals and exploration, but together with that they have downsides - e.g. there are a few provided documentations of how source code and generated data should be maintained and operated.\cite{johansson2014introduction} 


\subsection{Proposed Approach}

\subsection{Structure of the paper}